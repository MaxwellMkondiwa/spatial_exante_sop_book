% Options for packages loaded elsewhere
\PassOptionsToPackage{unicode}{hyperref}
\PassOptionsToPackage{hyphens}{url}
\PassOptionsToPackage{dvipsnames,svgnames,x11names}{xcolor}
%
\documentclass[
  letterpaper,
  DIV=11,
  numbers=noendperiod]{scrreprt}

\usepackage{amsmath,amssymb}
\usepackage{iftex}
\ifPDFTeX
  \usepackage[T1]{fontenc}
  \usepackage[utf8]{inputenc}
  \usepackage{textcomp} % provide euro and other symbols
\else % if luatex or xetex
  \usepackage{unicode-math}
  \defaultfontfeatures{Scale=MatchLowercase}
  \defaultfontfeatures[\rmfamily]{Ligatures=TeX,Scale=1}
\fi
\usepackage{lmodern}
\ifPDFTeX\else  
    % xetex/luatex font selection
\fi
% Use upquote if available, for straight quotes in verbatim environments
\IfFileExists{upquote.sty}{\usepackage{upquote}}{}
\IfFileExists{microtype.sty}{% use microtype if available
  \usepackage[]{microtype}
  \UseMicrotypeSet[protrusion]{basicmath} % disable protrusion for tt fonts
}{}
\makeatletter
\@ifundefined{KOMAClassName}{% if non-KOMA class
  \IfFileExists{parskip.sty}{%
    \usepackage{parskip}
  }{% else
    \setlength{\parindent}{0pt}
    \setlength{\parskip}{6pt plus 2pt minus 1pt}}
}{% if KOMA class
  \KOMAoptions{parskip=half}}
\makeatother
\usepackage{xcolor}
\setlength{\emergencystretch}{3em} % prevent overfull lines
\setcounter{secnumdepth}{5}
% Make \paragraph and \subparagraph free-standing
\ifx\paragraph\undefined\else
  \let\oldparagraph\paragraph
  \renewcommand{\paragraph}[1]{\oldparagraph{#1}\mbox{}}
\fi
\ifx\subparagraph\undefined\else
  \let\oldsubparagraph\subparagraph
  \renewcommand{\subparagraph}[1]{\oldsubparagraph{#1}\mbox{}}
\fi


\providecommand{\tightlist}{%
  \setlength{\itemsep}{0pt}\setlength{\parskip}{0pt}}\usepackage{longtable,booktabs,array}
\usepackage{calc} % for calculating minipage widths
% Correct order of tables after \paragraph or \subparagraph
\usepackage{etoolbox}
\makeatletter
\patchcmd\longtable{\par}{\if@noskipsec\mbox{}\fi\par}{}{}
\makeatother
% Allow footnotes in longtable head/foot
\IfFileExists{footnotehyper.sty}{\usepackage{footnotehyper}}{\usepackage{footnote}}
\makesavenoteenv{longtable}
\usepackage{graphicx}
\makeatletter
\def\maxwidth{\ifdim\Gin@nat@width>\linewidth\linewidth\else\Gin@nat@width\fi}
\def\maxheight{\ifdim\Gin@nat@height>\textheight\textheight\else\Gin@nat@height\fi}
\makeatother
% Scale images if necessary, so that they will not overflow the page
% margins by default, and it is still possible to overwrite the defaults
% using explicit options in \includegraphics[width, height, ...]{}
\setkeys{Gin}{width=\maxwidth,height=\maxheight,keepaspectratio}
% Set default figure placement to htbp
\makeatletter
\def\fps@figure{htbp}
\makeatother

\KOMAoption{captions}{tableheading}
\makeatletter
\@ifpackageloaded{bookmark}{}{\usepackage{bookmark}}
\makeatother
\makeatletter
\@ifpackageloaded{caption}{}{\usepackage{caption}}
\AtBeginDocument{%
\ifdefined\contentsname
  \renewcommand*\contentsname{Table of contents}
\else
  \newcommand\contentsname{Table of contents}
\fi
\ifdefined\listfigurename
  \renewcommand*\listfigurename{List of Figures}
\else
  \newcommand\listfigurename{List of Figures}
\fi
\ifdefined\listtablename
  \renewcommand*\listtablename{List of Tables}
\else
  \newcommand\listtablename{List of Tables}
\fi
\ifdefined\figurename
  \renewcommand*\figurename{Figure}
\else
  \newcommand\figurename{Figure}
\fi
\ifdefined\tablename
  \renewcommand*\tablename{Table}
\else
  \newcommand\tablename{Table}
\fi
}
\@ifpackageloaded{float}{}{\usepackage{float}}
\floatstyle{ruled}
\@ifundefined{c@chapter}{\newfloat{codelisting}{h}{lop}}{\newfloat{codelisting}{h}{lop}[chapter]}
\floatname{codelisting}{Listing}
\newcommand*\listoflistings{\listof{codelisting}{List of Listings}}
\makeatother
\makeatletter
\makeatother
\makeatletter
\@ifpackageloaded{caption}{}{\usepackage{caption}}
\@ifpackageloaded{subcaption}{}{\usepackage{subcaption}}
\makeatother
\ifLuaTeX
  \usepackage{selnolig}  % disable illegal ligatures
\fi
\usepackage{bookmark}

\IfFileExists{xurl.sty}{\usepackage{xurl}}{} % add URL line breaks if available
\urlstyle{same} % disable monospaced font for URLs
\hypersetup{
  pdftitle={Workflows for Spatial Exante Economic Analytics: Selected SOPs},
  pdfauthor={Maxwell Mkondiwa (with Jordan Chamberlin and CSISA-EiA Team},
  colorlinks=true,
  linkcolor={blue},
  filecolor={Maroon},
  citecolor={Blue},
  urlcolor={Blue},
  pdfcreator={LaTeX via pandoc}}

\title{Workflows for Spatial Exante Economic Analytics: Selected SOPs}
\author{Maxwell Mkondiwa (with Jordan Chamberlin and CSISA-EiA Team}
\date{Invalid Date}

\begin{document}
\maketitle

\renewcommand*\contentsname{Table of contents}
{
\hypersetup{linkcolor=}
\setcounter{tocdepth}{2}
\tableofcontents
}
\bookmarksetup{startatroot}

\chapter*{Preface}\label{preface}
\addcontentsline{toc}{chapter}{Preface}

\markboth{Preface}{Preface}

What is all this? Standard Operating Procedures (SOP) for a research
process which should be driven by a quest to discover the unknown seems
far-fetched. It is far-fetched indeed as it should be. Let alone for
spatial exante which depends very much on what the context demands. Our
purpose for these SOPs is to standardize the spatial exante process so
as to shorten the amount of time researchers spend on coding the common
building blocks of spatial exante workflows\hyperref[_ftn1]{{[}1{]}}.

The choice of which methods to showcase in the SOP reflects our taste
and level of knowledge. It also reflects our goal which is to provide a
prioritization assessment framework for potential investments.

We focus on three indicators for prioritization: yield gains, profit
gains and robustness to risk.

\hyperref[_ftnref1]{{[}1{]}} We develop the SOPs following advice from
Hollman et al (2020)
{[}https://journals.plos.org/ploscompbiol/article?id=10.1371/journal.pcbi.1008095{]}.

\textbf{Acknowledgements:} We thank CSISA and EiA for funding this work.
This draft manual was written by Maxwell Mkondiwa with support from
Jordan Chamberlin. It has not been reviewed thoroughly. All remaining
errors are ours.

\bookmarksetup{startatroot}

\chapter{Introduction}\label{introduction}

\section{Overview}\label{overview}

CSISA-EiA Spatial Exante Standard Operating Procedure (SpSOP) is a suite
of validated toolkits that use economic theory, econometrics, and
mathematical optimization models to target investments of different
agronomic innovations at disaggregated spatial scales, as well as
evaluating the returns on these investments. The returns are expressed
depending on the toolkit as yield gains, profit gains, probability of
getting a threshold level of yield gains or profit gains, yield or
profit risk, producer surplus, financial indicators like cost benefit
ratio and net present value, and willingness to pay measures. ~~

We start with exploratory toolkit which involves literature review, back
of the envelope calculations, and stakeholder
engagements\hyperref[_ftn1]{{[}1{]}}. In the standard procedures for
evaluating agricultural research (as described in Alston et al 1995),
this procedure involves using scoring and other short cut approaches.
This procedure is not a replacement for the other more objective and
data grounded procedures we discuss next. It is a starting point for
doing spatial exante work as it guides the nature and scope of toolkits
to use next.

The next toolkit in the system assesses whether there are substantial
economic gains for a farmer who is either risk agnostic or risk averse
to likely adopt the technology. This toolkit assumes that the farmers
are individually too small to affect the prices and quantities of other
farmers.

In following toolkit, we relax these assumptions to estimate the
producer and consumer surplus while considering farmers' demand and
supply price behaviour and equilibrium relationships. The two toolkits
presented assume that the technology already exists and that all that
remains is to increase its adoption.

In the final toolkit, we present a case of first ascertaining whether
there is adoption potential given spatially explicit endowments and
whether by adding attributes to the technology, farmers are then more
likely to adopt it. We then use that system to compute the economic
value of that adoption to the farmers.

How does one choose which toolkits to use? We present next a guide map
to choosing the analytical methods given the available data, expertise
and time.

\hyperref[_ftnref1]{{[}1{]}} Within CSISA and EiA, the CAPTAIN tool (now
called PAiCE) is a clear example of this toolkit.

\section{Guidelines to comprehensive spatial quantitative exante
toolkits}\label{guidelines-to-comprehensive-spatial-quantitative-exante-toolkits}

This paper has provides a list of spatial quantitative exante toolkits
that are used to prioritize potential investments. These toolkits mostly
use existing survey data or baseline data that most projects ideally
collect prior to implementation.

Figure 1 shows a guidemap to selecting which exante approach to use for
a selected study. We categorize the decision steps into four layers.
First, one has to conduct a literature review, back of the envelope
calculations of structural changes in the economy, and stakeholder
engagements. This layer needs to be done regardless of the comprehensive
approaches that are later used in the spatial exante assessment. It is
possible to stop and start implementing at this point if prior exante
studies were already conducted on the topic of interest and at a
sufficient scale. In the second layer, one gathers all the necessary
datasets required for the exante work. If there is no data, then instead
of scaling the interventions, it is best to work with stakeholders to
design on station, on-farm experiments, quantitative and qualitative
surveys to start gathering evidence to be used for exante.

If survey data exist already, then one needs to start with the spatial
profitability and risk assessment toolkit. In this tookit, the
researcher needs to ask if the technology in question is sufficiently
studied elsewhere such that there are already functional forms to use
the parametric approach. If not clear on this then, he/she may use the
causal ML based approaches. The researcher may use the spatial Bayesian
krigging approach if the targeting is to focus on locations to implement
including out of sample. He/she may consider the policy learning
optimization approach if he/she is interested in understanding the
indicators to use when partitioning who needs to be prioritized beyond
the spatial aspects.

If crop simulation and long-term experimental datasets are available,
one can use the spatial profitability and risk assessment toolkit as
well. But in addition, he/she may be interested in using the evidence
from these experiments to suggest new combinations of treatments that
should be tested or scaled beyond what is observed to be beneficial. For
that, the researcher can use the modern portfolio approaches (i.e.,
mean-variance optimization or mean-conditional value at risk
optimization). If instead he/she is just interested at recommending for
each grid the most robust practice for scaling, then he/she can use the
willingness to pay bounds approach which recommends the best practice
for any risk averse farmer to find it optimal to follow that strategy.

The spatial profitability assessment toolkit and the spatial
optimization toolkit will provide an individual assessment of the
benefits of. However, one may be interested to understand if this makes
sense socially as well given the demand and supply behavioral patterns
(i.e., elasticities). The equilibrium displacement modelling framework
also called economic surplus approach is the traditional way of
assessing producer and consumer surplus for the new agronomic
technology. This approach will utilize supply/demand elasticities, the
spatially explicit yield gains and cost reduction estimates and the
assumed adoption trends to evaluate the returns on investments at a
disaggregated level.

What if the new technology is just a variant of an old technology. For
example, a new mix of herbicides or a new variety. In most cases, these
are not widely adopted to warrant a spatial profitability assessment
using observational data. In addition, the performance in agronomic
trials will likely be misleading because farmers have not yet learned
how to use the technology appropriately. How do we predict whether this
new technology is will be adopted and that it's worth investing in. For
this, the researcher needs to consider the structural differentiated
agronomy toolkit which uses characteristics of the technology and
locational characteristics (including farmer demographics) to predict
whether the new technology embodying particular traits is worth
investing.

As it can be seen from the guidemap, these toolkits can be used in
tandem because they give different insights into the likely benefits of
investing in a particular technology. In addition, these approaches are
not exhaustive. Each of the workflows we have presented have several
variants and complementary methods that researchers can explore.

\begin{figure}[H]

{\centering \includegraphics{Guidemap.png}

}

\caption{Figure 1: Guidemap to selecting spatial exante approaches}

\end{figure}%

\section{Target audience}\label{target-audience}

The toolkits can be used by researchers, project managers, and students
to conduct exante analyses that guide investments. Our target audience
are researchers who are tasked to conduct spatially explicit exante
analyses. We thus assume that the researchers have used most of these
tools before or are at least aware of them. For those not familiar with
the methods, we encourage the reader to go through the suggested
references.

\section{Stylized example: Sowing date use
case}\label{stylized-example-sowing-date-use-case}

We focus on sowing date use case being implemented in India as an
exemplar for the exante analytics presented in this SOP. However, we
have applied the same techniques for other use cases including: (1)
Herbicide integrated weed management in wheat, (2) multiple irrigations
in wheat (irrigation scheduling) and short and long duration wheat
varieties (varietal choice).

\section{Replication materials}\label{replication-materials}

The replication materials for all the toolkits can be accessed on EiA
Exante github page: \url{https://github.com/EiA2030-ex-ante}. We use
publicly available datasets as such once the repository of interest is
cloned, one should be able to replicate all the results in this SOP.~

\bookmarksetup{startatroot}

\chapter{Spatial exante profitability and risk
toolkit}\label{spatial-exante-profitability-and-risk-toolkit}

\section{Spatial parametric production function approach with
risk}\label{spatial-parametric-production-function-approach-with-risk}

\textbf{Purpose:} The conventional approach is to estimate a parametric
production function (also called crop response function), then use
profit maximization or it's dual cost minimization to identify optimal
demand for the associated technology. If risk is considered important,
the traditional approach is to assume a quadratic utility function and
use mean-variance approaches to assess optimal choices under risk
aversion preferences (e.g., Just-Pope production function or moments
productions). To use this approach for spatial exante, we use spatial
Bayesian models for point-referenced data (Note: spatial econometric
approaches can also be used for this extension to the traditional
model).

\textbf{Advantages}

\begin{itemize}
\tightlist
\item
  Simple to use with standard econometric approaches (e.g., OLS).
\end{itemize}

\textbf{Disadvantages}

\begin{itemize}
\tightlist
\item
  Difficult to identify the appropriate functional form and the results
  are largely dependent on this choice.
\end{itemize}

\textbf{Stylized use case: Are sowing date advisories risk proof?}

We use CSISA-KVK trial data to understand whether early sowing of wheat
and planting of long duration wheat varieties increase mean yield and
reduce risk. Workflow Figure 4 shows a workflow for the spatial
parametric production function approach. This is categorized into four
steps. First, one estimates a production risk function model using
either the residual based (e.g., Just-Pope production function) or the
moments-based approach using ordinary least squares approach. If there
are concerns with endogeneity, then one can correct for these using the
instrumental variables approach or other quasi-experimental methods. The
simple approach we recommend is using Lewbel (2012) approach. Third, to
make the estimate spatially explicit, we recommend using a spatially
varying coefficient model to get estimates for each pixel in area of
interest. Finally, one can use input and output prices to create
economic indicators of interest.

\begin{figure}[H]

{\centering \includegraphics{spatial_JP_risk_model.png}

}

\caption{Figure 4: Workflow for spatial risk production function}

\end{figure}%%
\begin{figure}[H]

{\centering \includegraphics{JP_model_figure.png}

}

\caption{Figure 5: Just-Pope production risk model}

\end{figure}%%
\begin{figure}[H]

{\centering \includegraphics{Moments_based_risk_figure.png}

}

\caption{Figure 6: Moments based production risk model}

\end{figure}%

\textbf{\emph{Replication materials:
\url{https://eia2030-ex-ante.github.io/SpatialParametricProduction_Risk_Model/}}}

\textbf{\emph{Key references:}}

Antle, J.M. 1983. ``Testing the stochastic structure of production: A
flexible moment-based approach''. \emph{Journal of Business and Economic
Statistics} 1(3): 192-201. Doi: 10.1080/07350015.1983.10509339.

Antle, J.M. 2010. ``Asymmetry, partial moments and production risk.''
\emph{American Journal of Agricultural Economics 92(5):} . Doi:
\url{https://doi.org/10.1093/ajae/aaq077}.

Di Falco, S., Chavas, J., and Smale, M. 2007. ``Farmer management of
production risk on degraded lands: the role of wheat variety diversity
in the Tigray region, Ethiopia.'' \emph{Agricultural Economics} 36:
147-156. Doi: ~\url{https://doi.org/10.1111/j.1574-0862.2007.00194.x}.

Di Falco, S., and Chavas, J. 2009. ``On crop biodiversity, risk
exposure, and food security in the highlands of Ethiopia''.
\emph{American Journal of Agricultural Economics} 91(3): 599-611. Doi:
\url{https://doi.org/10.1111/j.1467-8276.2009.01265.x}.

\section{Causal ML and spatial probabilistic assessment
model}\label{causal-ml-and-spatial-probabilistic-assessment-model}

\textbf{\emph{Purpose:}} In some cases, the farmer is not only
interested in shifting to a technology that gives the highest yield
gains, but also the one that has the highest chance of giving him/her
yields beyond a particular threshold.

\textbf{\emph{Advantages}}

The spatial probabilistic approach adds value under the following
circumstances:

\begin{itemize}
\item
  One is interested in segmentation of zones of opportunities.
\item
  One is interested in threshold probabilities as measures of
  uncertainty.
\end{itemize}

\textbf{\emph{Disadvantages}}

\begin{itemize}
\tightlist
\item
  The spatial Bayesian models are computationally expensive especially
  for large N data and can take many weeks to produce results. This can
  be resolved by using High Performance Computers.
\end{itemize}

\textbf{\emph{Stylized use case: Where to target sowing date advisories
that achieve yield gains beyond a particular threshold?}}

A farmer requires a substantial yield gain to change from the
conventional behaviour. In recommending planting date changes, it is
therefore important to provide the confidence we have that the farmer
will likely attain yields higher than that threshold. A probabilistic
assessment approach allows this through a threshold probability---the
probability that a farmer in that location will achieve yield gains
above the threshold.

\textbf{\emph{Input data requirements:}} The approach requires
geo-referenced farm plots with attendant yield and traditional
production variables (e.g., fertilizer, weed management, e.t.c).

\textbf{\emph{Toolkit workflow}}

This toolkit is implemented by following the steps shown in the figure
7.

\begin{figure}[H]

{\centering \includegraphics{spatial_probabilistic_assessment_toolbox.png}

}

\caption{Figure 7: spatial probabilistic assessment toolbox}

\end{figure}%

\textbf{\emph{Stylized outputs}}

Using this toolkit, we see in Figure 8 that farmers in much of the area
of interest (Bihar) would find early planting of wheat most beneficial
and have a probability of getting an additional 100kg/ha due to early
sowing alone.

\begin{figure}[H]

{\centering \includegraphics{spatial_prob_figure.png}

}

\caption{Figure 8: Stylized output for spatial probabilistic assessment
showing probability of yield gains of above 100kg/ha with early planting
of wheat (i.e., before 21st Nov)}

\end{figure}%

\textbf{\emph{Replication materials:
\url{https://github.com/EiA2030-ex-ante/Spatial_probabilistic_targeting}}}

\textbf{\emph{Key references:}}

Athey, S., Tibshirani, J., and Wager, S. 2019. ``Generalized Random
Forests.'' \emph{The Annals of Statistics} 47(2): 1148-1178. Doi:
10.1214/18-AOS1709.

McCullough, E.B., Quinn, J.D., Simons, A.M. 2020. ``Profitability of
climate-smart soil fertility investment varies widely across sub-Saharan
Africa.'' \emph{Nature Food} 3:275-285. Doi:
\url{https://doi.org/10.1038/s43016-022-00493-z}.

\section{Causal ML and policy learning optimization
model}\label{causal-ml-and-policy-learning-optimization-model}

\textbf{\emph{Purpose:}} To make individualized or personalized
recommendations from observational data in a data-driven manner using
causal machine learning frameworks.

\textbf{\emph{Advantages}}

\begin{itemize}
\tightlist
\item
  Data-driven approach of recommending alternatives without making
  functional form assumptions. This is especially useful for
  agricultural inputs for which we do not have a clear functional form
  e.g., irrigation, sowing dates.
\end{itemize}

\textbf{\emph{Disadvantages}}

\begin{itemize}
\tightlist
\item
  It requires enough sample sizes for each of the options being
  compared. This mean that for new innovations which have not been
  extensively adopted, this approach would not be beneficial.
\end{itemize}

\textbf{\emph{Stylized use case: Targeting sowing date advisories to
individual farmers}}

While sowing date and many other recommendations are made on the basis
of climatic, biophysical and economic aspects, there may be several
individual level reasons for not following with the recommendation,
e.g., family members are busy with other duties during those weeks. We
propose a robust methodology that rests on causal machine learning and
policy learning to make recommendations that are the most beneficial for
each individual farmer.

\textbf{\emph{Input data requirements:}} The data required is the same
as for any conventional production function or impact assessment. These
include yield, agronomic management variables (e.g., fertilizer
applied), socio-economic variables, and input and output prices. One
however, needs enough sample sizes for the treatment and control groups
therefore the method works only for a technology which has been widely
adopted.

\textbf{\emph{Toolkit workflow}}

Figure 9 shows a step-by-step workflow for implementing the policy
learning optimization model.

\begin{figure}[H]

{\centering \includegraphics{Causal_ML_prioritization_workflow.png}

}

\caption{Figure 9: Workflow for causal ML and policy learning
optimization}

\end{figure}%

\textbf{\emph{Stylized outputs}}

The output of steps 1 to 3 in the workflow are the individual level
estimates of the yield gains from the proposed agronomic innovation.
Figure 10 shows the distribution of yield gains to early sowing.
Everyone in the sample would get a positive yield gain if they advance
their planting strategy as compared to sowing after 16 December.~ The
highest yield gains are with planting before 10\textsuperscript{th}
November. However, the results in this figure do not prescribe a
recommendation for that farmer. To prescribe a recommendation, we need
to assume some objective function of the farmer. Policy learning uses
minimum regret as an objective function to prescribe best practice for
each farmer. Figure 11 then shows the transition matrix from status quo
to proposed agronomic practice for each individual farmer.

\begin{figure}[H]

{\centering \includegraphics{Causal_ML_cates_distribution_figure.png}

}

\caption{Figure 10: Distribution of conditional average treatment
effects of wheat yield gains to early sowing from multi-armed causal ML
model}

\end{figure}%%
\begin{figure}[H]

{\centering \includegraphics{Transition_matrix_figure.png}

}

\caption{Figure 11: Transition matrix from status quo (as of 2019) to
optimal allocations}

\end{figure}%

\textbf{\emph{Replication materials:
\url{https://eia2030-ex-ante.github.io/causal_RF_targeting/}}}

\textbf{\emph{Key references}}

Athey, S., and Wager, S. 2021. ``Policy learning with observational
data''. \emph{Econometrica}. Url:
\url{https://onlinelibrary.wiley.com/doi/abs/10.3982/ECTA15732}.

\bookmarksetup{startatroot}

\chapter{Spatial optimization toolkit: Computational risk-return
modeling}\label{spatial-optimization-toolkit-computational-risk-return-modeling}

\section{Mean-Variance (EV) and Mean-Conditional Value at Riskk (CVaR)
modern portfolio theory
optimization}\label{mean-variance-ev-and-mean-conditional-value-at-riskk-cvar-modern-portfolio-theory-optimization}

\textbf{Purpose:} Mean-variance analysis seeks to maximize returns at
the minimum risk (or variance). The approach was introduced by Harry
Markowitz to identify efficient diversification options for investments.

\textbf{\emph{Advantages}}

\begin{itemize}
\tightlist
\item
  Allows selection of multiple alternatives beyond combinations observed
  in the data
\end{itemize}

\textbf{\emph{Disadvantages}}

\begin{itemize}
\tightlist
\item
  Focuses only two moments (mean and variance) yet other moments of the
  distribution may also matter.
\end{itemize}

\textbf{\emph{Stylized use case: Optimal sowing date and variety
combinations}}

We use CSISA-KVK trial data to demonstrate the approach. The agronomic
trials cover a 5-year period in 8 districts in the Indian state of
Bihar.

\textbf{\emph{Input data requirements:}} This is an outcome-based risk
assessment requiring yield or profits data for multiple years for the
same site.

\textbf{\emph{Toolkit workflow}}

Mean-Variance optimization requires only the outcome variable for
multiple realizations and portfolio choices. We then use quadratic
optimization to identify the frontier and optimal weights indicating the
amount of land or resources that should be devoted to particular
choices. ~

\begin{figure}[H]

{\centering \includegraphics{EV_CVaR_workflow.png}

}

\caption{Figure 12: Mean-variance or mean-conditional value at risk
(CVaR) workflow}

\end{figure}%

\textbf{\emph{Stylized outputs}}

Using a state level E-V optimization model of wheat yields, we find that
HD-2967 sown before 10\textsuperscript{th} November gives the highest
returns and a risk neutral farmer would find it most beneficial.

\begin{figure}[H]

{\centering \includegraphics{sowing_date_varietal_yield_frontier.png}

}

\caption{Figure 13: Planting date-varietal yield frontier for Bihar,
2016-2021.}

\end{figure}%%
\begin{figure}[H]

{\centering \includegraphics{optimal_weights.png}

}

\caption{Figure 14: Optimal weights for planting date-variety yield
frontier, 2016-2021.}

\end{figure}%

\textbf{\emph{Replication materials:
\url{https://eia2030-ex-ante.github.io/Risk_modern_portfolio_theory_EV_model/}}}

\section{Willingness to pay bounds for second order stochastic dominance
approach}\label{willingness-to-pay-bounds-for-second-order-stochastic-dominance-approach}

\textbf{Purpose:} The commonly used risk measures focus on central
moments (e.g., variance, conditional value at risk, skewness) of the
distribution. Yield distributions overtime and space are however more
complicated such that one may need to consider the whole distribution
when evaluating which agronomic practice will likely work where and
when. The use of stochastic dominance especially second order stochastic
dominance allows the relationship between the cumulative distribution
function of the outcome and the expected utility maximization behaviour
under risk aversion. A computational approach developed by Hurley et al
(2018) allows one to compute willingness to pay lower and upper bounds
for a new technology to second order stochastically dominate an old
practice such that any risk averse farmer will choose the new
technology.

\textbf{\emph{Advantages}}

\begin{itemize}
\tightlist
\item
  Unlike mean-variance optimization, this optimization strategy
  considers distributional comparisons
\end{itemize}

\textbf{\emph{Disadvantages}}

\begin{itemize}
\item
  Computationally expensive especially when implementing across a large
  area of interest.
\item
  The comparisons are pairwise thereby requiring many combinations to
  come up with the best alternative for each pixel.
\item
  Difficult to apply with survey or agronomic datasets are it requires
  long timeseries. However, it is possible to implement the approach
  with monte-carlo simulated survey or agronomic trial datasets.
\end{itemize}

\textbf{\emph{Stylized use case: Where to target sowing date
advisories?}}

We use gridded crop growth simulation model results to identify
scenarios that would be agronomically and economically beneficial even
for a risk averse farmer.

\textbf{\emph{Input data requirements}}

For the spatial exante (economic) component of the model, one only needs
gridded crop simulation results for each of the scenarios.

\textbf{\emph{Toolkit workflow}}

Figure 15 shows the workflow for implementation the computational second
order stochastic dominance analysis.

\begin{figure}[H]

{\centering \includegraphics{sosd_workflow.png}

}

\caption{Figure 15: Risk optimization using second order stochastic
dominance}

\end{figure}%

\textbf{\emph{Stylized outputs}}

\begin{figure}[H]

{\centering \includegraphics{system_profit_sosd_map.png}

}

\caption{Figure 16: Willingness to pay based on partial profits for
planting date scenario in comparison to fixed date with long duration
rice variety strategy}

\end{figure}%%
\begin{figure}[H]

{\centering \includegraphics{Optimal_rice_planting_strategy.png}

}

\caption{Figure 17: Robust and optimal rice planting date strategy}

\end{figure}%

\textbf{\emph{Replication materials:
\url{https://github.com/AntonUrfels/econ-sims-bihar}}}

\textbf{\emph{Key references:}} For more methodological details of the
approach, readers are referred to Hurley et al (2018).

Hurley, T., Koo, J., and Tesfaye, K. 2018. ``Weather risk: how does it
change the yield benefits of nitrogen fertilizer and improved maize
varieties in sub-Saharan Africa?'' \emph{Agricultural Economics} 49:
711-723. Doi: 10.1111/agec.12454.

\bookmarksetup{startatroot}

\chapter{Spatial economic surplus and return on investments (ROI)
toolkit}\label{spatial-economic-surplus-and-return-on-investments-roi-toolkit}

\section{Discounted cash flow (DCF) economic surplus framework
{[}Incomplete{]}}\label{discounted-cash-flow-dcf-economic-surplus-framework-incomplete}

\textbf{Purpose:} Economic surplus framework is the most used approach
in agricultural economics to evaluate the agricultural research
benefits.

In recent years, scholars have also suggested the use of real options
approach which then helps in valuing the time to wait and dynamic
complexities appropriately.

\textbf{\emph{Why?}}

·~~~~~~~ Relies on economic theory especially demand and supply as well
as welfare economics

\textbf{\emph{Why not?}}

·~~~~~~~ Given data requirements (e.g., on elasticities), the analyses
are done at aggregate level (e.g., country level).

\textbf{\emph{Stylized use case: Would sowing date advisories pay?}}

We estimate the returns on investments of early sowing advisories using
the economic surplus approach.

\textbf{Input data requirements:} Surplus analysis approach requires
data on area, yield , and output data. It also requires data on percent
of area under the new technology, the supply and demand elasticities,
yield gain due to the technology , price data and cost change data due
to the technology.

\textbf{\emph{Toolkit workflow}}

\includegraphics{spatial_econ_surplus_figure.png}

\bookmarksetup{startatroot}

\chapter{Structural differentiated agronomy
toolkit}\label{structural-differentiated-agronomy-toolkit}

\section{Targeting research, adoption and impacts of new innovations--
hybrid characteristics-induced innovation model
toolkit}\label{targeting-research-adoption-and-impacts-of-new-innovations-hybrid-characteristics-induced-innovation-model-toolkit}

\textbf{\emph{Why or why not a differentiated farming system (or
agronomy) framework?}}

A differentiated farming system approach focuses on identifying
characteristics of the farming systems which farmers would find
attractive. The goal is then to estimate the willingness of the farmers
to pay for these farming system characteristics.

The advantage of doing this is that:

·~~~~~~~ It allows one to predict the adoption of new agronomic
innovations

·~~~~~~~ It helps to assess the welfare impacts of many varieties of a
technology unlike the traditional approach which treats them as separate
inputs or goods.

\textbf{\emph{Stylized use case: Where to target sowing date
advisories?}}

We use the differentiated farming systems approach by analyzing the
segmentation of sowing dates and drivers associated with farmers choice
of which planting dates to sow their wheat.

\textbf{\emph{Input data requirements}}

\textbf{\emph{Toolkit workflow}}

\includegraphics{differentiated_agronomy_workflow.png}

\textbf{Selected results}

\includegraphics{OLS_estimates_diff_Agron.png}

\bookmarksetup{startatroot}

\chapter{Conclusion}\label{conclusion}

This paper has discussed toolkits for conducting spatial exante analyses
using a case study of sowing date advisories. These toolkits are
selected to showcase reproducible workflows that can be applied for
other agronomic practices and in other areas of interest. In on-going
research, we are also using mathematical optimization approaches,
agent-based modelling approaches, knowledge guided machine learning and
spatial multi-criteria approaches for targeting and prioritizing
agricultural innovations. We believe automating research around these
methods will allow agronomists and other stakeholders to quickly make
scientific discoveries and respond to ever changing agronomic
environments due to climate variability and change.

\bookmarksetup{startatroot}

\chapter{References}\label{references}

Alston, J.M., Norton, G.W., and Pardey, P. 1995. ``Science under
scarcity: Principles and practice for agricultural research evaluation
and priority setting''. CAB International. See Chapter 7 (pp.~463-498).

Antle, J.M. 1983. ``Testing the stochastic structure of production: A
flexible moment-based approach''. Journal of Business and Economic
Statistics 1(3): 192-201. Doi: 10.1080/07350015.1983.10509339.

Antle, J.M. 2010. ``Asymmetry, partial moments and production risk.''
American Journal of Agricultural Economics 92(5): . Doi:
https://doi.org/10.1093/ajae/aaq077.

Athey, S., Tibshirani, J., and Wager, S. 2019. ``Generalized Random
Forests.'' The Annals of Statistics 47(2): 1148-1178. Doi:
10.1214/18-AOS1709.

Athey, S., and Wager, S. 2021. ``Policy learning with observational
data''. Econometrica. Url:
https://onlinelibrary.wiley.com/doi/abs/10.3982/ECTA15732.

Binswanger, H.P. 1986. ``Evaluating Research System Performance and
Targeting Research in Land-abundant Areas of Sub-Saharan Africa.'' World
Development 14(4): 469-475. Doi:
https://doi.org/10.1016/0305-750X(86)90063-X.

Di Falco, S., Chavas, J., and Smale, M. 2007. ``Farmer management of
production risk on degraded lands: the role of wheat variety diversity
in the Tigray region, Ethiopia.'' Agricultural Economics 36: 147-156.
Doi: https://doi.org/10.1111/j.1574-0862.2007.00194.x.

Di Falco, S., and Chavas, J. 2009. ``On crop biodiversity, risk
exposure, and food security in the highlands of Ethiopia''. American
Journal of Agricultural Economics 91(3): 599-611. Doi:
https://doi.org/10.1111/j.1467-8276.2009.01265.x.

Hurley, T., Koo, J., and Tesfaye, K. 2018. ``Weather risk: how does it
change the yield benefits of nitrogen fertilizer and improved maize
varieties in sub-Saharan Africa?'' Agricultural Economics 49: 711-723.
Doi: 10.1111/agec.12454.

Lee, D.R., Edmeades, S., Nys, E., McDonald, A., Janssen, W. 2014.
``Developing local adaptation strategies for climate change in
agriculture: A priority-setting approach with application to Latin
America''. Global Environmental Change 29: 78-91. Doi:
https://doi.org/10.1016/j.gloenvcha.2014.08.002.

McCullough, E.B., Quinn, J.D., Simons, A.M. 2020. ``Profitability of
climate-smart soil fertility investment varies widely across sub-Saharan
Africa.'' Nature Food 3:275-285. Doi:
https://doi.org/10.1038/s43016-022-00493-z.



\end{document}
